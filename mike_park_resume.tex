\documentclass{resume}

\usepackage{amssymb}
\usepackage{url}

\renewcommand{\categoryfont}{\sc}

%
% set the space used for category titles here:
% use the same value for oddsidemargin and marginparwidth [the latter 
% 		will be reset to account for marginparsep]
% 
\setlength{\oddsidemargin}{1.1in}
\setlength{\marginparwidth}{1.1in}
% 
% calculate other dimensions [textwidth and evensidemargin] 
% in function of oddsidemargin and marginparwidth: 
% would be nicer to put in the class file...
%
\addtolength{\marginparwidth}{-\marginparsep}
 \setlength{\evensidemargin}{\oddsidemargin}
 \setlength{\textwidth}{\paperwidth}
 \addtolength{\textwidth}{-2in}
 \addtolength{\textwidth}{-2\oddsidemargin}
 \addtolength{\textwidth}{\marginparwidth}
 \addtolength{\textwidth}{\marginparsep}
%
%
\setlength{\topmargin}{-0.5in}
%
%
\renewcommand{\labelcitem}{$\rhd$}
\renewcommand{\labelitemi}{$\cdot$}
\newcommand{\first}{$1^{\mbox{\scriptsize st}}$\ }
\newcommand{\second}{$2^{\mbox{\scriptsize nd}}$\ }
\newcommand{\third}{$3^{\mbox{\scriptsize rd}}$\ }

\author{~~~~~~Michael Andrew Park}
% ------ Address --------------------------------------------------------

\address{
         Computational AeroSciences Branch\\
	 NASA Langley Research Center\\
         Mail Stop 128\\
	 Hampton, VA 23508\\
	 Office +1 (757) 864-6604\\
	 \mbox{\small\tt Mike.Park@NASA.gov}
        }{
	4608 Colonial Avenue\\
	Norfolk, VA 23508\\
	Mobile +1 (757) 535-6675\\
	 \mbox{\small\tt MikePark.rb@Gmail.com}}

\begin{document}
\maketitle

% ------- Education ---------------------------------------------------

\begin{category}{Education}
\citem{Massachusetts Institute of Technology}, Cambridge, MA.\\
Ph.D. in Computational Fluid Dynamics, Department of Aeronautics and Astronautics,  GPA 4.66 of 5.00, September 2008.
\citem{NASA Langley / George Washington University}, Hampton, VA.\\
Joint Institute for the Advancement of Flight Sciences
M.S., Aeronautical Engineering, GPA 3.28 of 4.00, August 2000.
\citem{University of Southern California}, Los Angeles, CA.\\
B.S., Aerospace Engineering, GPA 3.48 of 4.00, May 1998.
\end{category}

% --------- Research ----------------------------------------------------

\begin{category}{Research interests}
\citemnobullet 
Computational Fluid Dynamics (CFD) flow and adjoint
solver development for analysis, solution adaptation, and design;
anisotropic grid adaptation mechanics;
high-performance computing;
CFD application;
sonic boom prediction;
collaborative software development.
\end{category}

\begin{category}{Skills}
\citemnobullet 
C, Fortran, MATLAB, Ruby, Git, Subversion, GNU Autotools, \LaTeX.
\end{category}

\begin{category}{Work Experience}
\citem{Research Scientist}, NASA Langley Computational AeroSciences Branch;
September 2000--present.
Implemented a three-dimensional output (adjoint) based error
estimation and adaptation scheme in FUN3D.%
\footnote{\url{http://fun3d.larc.nasa.gov}, an unstructured CFD simulation tool with adjoint solver capable of analysis, design, and grid adaptation
across all flow speed regimes} 
Developed a parallel, anisotropic grid adaptation
library that is
developed test-first and contains a direct link to CAD geometry.
Implemented Message Passing Interface (MPI) communication in FUN3D.
Software manager for FUN3D and established the FUN3D distributed version 
control system with automated quarantine build and test
for collaborative team software development.
Supports numerous FUN3D users in government, industry, and academia with CFD
consultation for
sonic boom prediction, propulsion effects, ground vehicle performance,
and grid adaptation.
Provided the summary and statistical analysis of the
AIAA Sonic Boom Predication Workshops as Co-chair,
participant, and technical lead. 
Participated in the AIAA Drag Prediction,
Supersonic Shock-Boundary Layer Interaction,
and High Lift Prediction Workshops with output-adaptive grid methods.
Performed turbulent eddy resolving calculations as part of the F-16XL
Cranked-Arrow Wing Aerodynamics Project International (CAWAPI).

\citem{Research Assistant} Langley Multidisciplinary Optimization Branch;
September 1998--August 2000.
Applied the ADIFOR (Automatic Differentiation in FORTRAN) tool to
CFL3D, a structured grid, Navier-Stokes CFD code
for stability and control derivatives.

\citem{Co-op Flight Test Engineer} 
NASA Dryden Flight Research Center Aerodynamics, Propulsion, and Controls
Branches; September 1995--August 1997. 
Improved air data reconstruction for F-18 HARV (High Alpha
Research Vehicle). Supported the Linear Aerospike
SR-71 Experiment (LASRE) with supersonic wind tunnel testing and
computational flow solutions for stability and control prediction.
Programmed a graphical modern control law design tool in MATLAB with
application to the Hyper-X hypersonic free flyer.
Explored alternative trajectories for
Hyper-X booster and free flyer separation.


\end{category}

% -------- Reference --------------------------------------------

\begin{category}{References, Publications} 
\citemnobullet Available on request.
\end{category}

\end{document}
